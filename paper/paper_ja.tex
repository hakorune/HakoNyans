\documentclass[a4paper,11pt]{ltjsarticle}
\usepackage{graphicx}
\usepackage{booktabs}
\usepackage{geometry}
\usepackage{float}
\usepackage{url}
\geometry{margin=24mm}
\graphicspath{{figures/}}

\title{HakoNyans: \\
PNG 代替を目標とした並列デコード志向画像コーデックの実装評価}
\author{HakoNyans Project}
\date{2026-02-11}

\begin{document}
\maketitle

\begin{abstract}
本稿では、並列デコード性能を主軸に設計した画像コーデック HakoNyans を評価する。
評価は二層で実施した。
(1) 定性比較: Anime/UI/Photo の 3 画像で原画像・JPEG・HKN を並列比較。
(2) 定量比較: PNG lossless と HKN lossless、JPEG と HKN lossy のサイズ・画質・速度を比較。
評価時点の再現基準コミットは \texttt{b4a16e8} である。
\end{abstract}

\section{目的}
HakoNyans の現時点の狙いは、最高圧縮率の競争ではなく、
実装の単純性を保ちながらデコードの並列性を確保することである。
特に、PNG 代替としての位置づけを明確化するため、
lossless 比較では PNG を主基準とした。

\section{方式概要}
\subsection{エンコード/デコードの流れ}
HakoNyans はタイル内 8x8 ブロックを基本単位とする。
エンコード側はブロックを解析してモードを選択し、最終的に NyANS-P で符号化する。
デコード側はヘッダとチャンクを順に復号し、ブロックを独立に再構成する。

\begin{itemize}
  \item Encode: Colorspace変換 $\rightarrow$ ブロック解析 $\rightarrow$ モード選択 $\rightarrow$ エントロピー符号化
  \item Decode: チャンク復号 $\rightarrow$ モード別再構成 $\rightarrow$ Colorspace逆変換
\end{itemize}

\subsection{lossless モード決定}
lossless 経路では、各 8x8 ブロックについて
\texttt{COPY / PALETTE / FILTER / TILE\_MATCH4} を候補とし、
推定ビットコストが最小のモードを選ぶ。

\begin{equation}
m^\star = \arg\min_{m \in \mathcal{M}} \hat{R}(m)
\end{equation}

ここで $\hat{R}(m)$ は mode type、参照オフセット、パレットサイズ、残差分布などから計算する近似レートである。
PHOTO 系では MED を有効化し、UI/Anime では Copy/Palette の寄与を優先する。

\subsection{lossy 経路(DCT + Band-group CDF + CfL)}
lossy 経路は YCbCr 変換後に DCT・量子化を行い、
DC と AC(LOW/MID/HIGH)を分離して Band-group CDF で符号化する。
クロマ予測には CfL を用い、ブロックごとに適用可否を判定する。

CfL の復号予測は次式で与える。
\begin{equation}
\hat{c} = \left(a \cdot (y - 128) + 32\right) \gg 6 + b
\end{equation}

ここで $a$ は量子化係数(Q6)、$b$ はオフセットである。
エンコード時は同じ量子化後予測式で誤差を評価し、適用時の色ずれを防ぐ。

\subsection{計算量と並列性}
主要処理は画素数 $N$ に対して概ね線形時間であり、復号は探索を伴わない。
ブロック再構成をタイル単位・行単位に分割することで並列実行しやすい。
この設計により、圧縮率の改善を主にエンコーダ側の工夫で行いながら、
デコーダ側の構造は単純に保つ方針を採る。

\section{評価設定}
\subsection{データセット}
本稿の可視化は 3 カテゴリを使用した。
\begin{itemize}
  \item Anime: \texttt{anime\_sunset.ppm}
  \item UI: \texttt{vscode.ppm}
  \item Photo: \texttt{nature\_01.ppm}
\end{itemize}

数値評価は UI/Anime/Photo の 7 画像で実施した。

\subsection{比較対象}
\begin{itemize}
  \item lossless: PNG vs HKN lossless
  \item lossy: JPEG(Q75/Q90) vs HKN(Q75/Q90, 4:4:4, CfL=1)
\end{itemize}

\subsection{指標}
\begin{itemize}
  \item サイズ: KB
  \item 画質: PSNR(dB)
  \item 速度: encode/decode ms
\end{itemize}

\section{定性結果(3カテゴリ)}
\begin{figure}[H]
  \centering
  \includegraphics[width=\linewidth]{qual_anime_crop_strip.png}
  \caption{Anime: Original / JPEG Q75 / HKN Q75(4:4:4 CfL=1) のクロップ比較}
\end{figure}

\begin{figure}[H]
  \centering
  \includegraphics[width=\linewidth]{qual_ui_crop_strip.png}
  \caption{UI: Original / JPEG Q75 / HKN Q75(4:4:4 CfL=1) のクロップ比較}
\end{figure}

\begin{figure}[H]
  \centering
  \includegraphics[width=\linewidth]{qual_photo_crop_strip.png}
  \caption{Photo: Original / JPEG Q75 / HKN Q75(4:4:4 CfL=1) のクロップ比較}
\end{figure}

\section{定量結果}
\subsection{PNG vs HKN lossless}
\begin{table}[H]
  \centering
  \caption{PNG と HKN lossless の比較(UI/Anime/Photo)}
  \resizebox{\linewidth}{!}{\begin{tabular}{llrrrrr}
\toprule
Image & Category & PNG (KB) & HKN (KB) & HKN/PNG & PNG Dec (ms) & HKN Dec (ms) \\
\midrule
browser & UI & 10.0 & 21.5 & 2.15x & 12.31 & 13.14 \\
vscode & UI & 11.4 & 50.0 & 4.39x & 12.05 & 15.21 \\
terminal & UI & 9.7 & 27.9 & 2.88x & 11.42 & 14.19 \\
anime\_girl & Anime & 9.0 & 37.6 & 4.18x & 11.40 & 14.02 \\
anime\_sunset & Anime & 10.4 & 39.9 & 3.84x & 13.25 & 14.62 \\
nature\_01 & Photo & 1251.4 & 919.1 & 0.73x & 16.37 & 42.11 \\
nature\_02 & Photo & 1412.6 & 992.6 & 0.70x & 19.51 & 48.59 \\
\bottomrule
\end{tabular}
}
\end{table}

\subsection{JPEG vs HKN lossy (Q75)}
\begin{table}[H]
  \centering
  \caption{カテゴリ平均(Q75)}
  \begin{tabular}{llrrrr}
\toprule
Category & Codec & Avg Size (KB) & Avg PSNR (dB) & Enc (ms) & Dec (ms) \\
\midrule
Anime & HKN\_q75\_444\_cfl1 & 139.5 & 39.36 & 475.3 & 32.6 \\
Anime & JPEG\_q75 & 21.1 & 42.99 & 8.0 & 7.1 \\
Anime & PNG\_lossless & 10.1 & 99.00 & 38.1 & 43.5 \\
Photo & HKN\_q75\_444\_cfl1 & 989.2 & 34.51 & 548.5 & 48.8 \\
Photo & JPEG\_q75 & 189.7 & 74.20 & 11.0 & 9.8 \\
Photo & PNG\_lossless & 1385.1 & 99.00 & 190.0 & 58.4 \\
UI & HKN\_q75\_444\_cfl1 & 448.7 & 40.09 & 491.5 & 40.0 \\
UI & JPEG\_q75 & 122.6 & 38.67 & 10.1 & 8.4 \\
UI & PNG\_lossless & 4.4 & 99.00 & 26.4 & 36.8 \\
\bottomrule
\end{tabular}

\end{table}

\subsection{JPEG vs HKN lossy (Q90)}
\begin{table}[H]
  \centering
  \caption{カテゴリ平均(Q90)}
  \begin{tabular}{llrrrr}
\toprule
Category & Codec & Avg Size (KB) & Avg PSNR (dB) & Enc (ms) & Dec (ms) \\
\midrule
Anime & HKN\_q90\_444\_cfl1 & 225.6 & 40.06 & 480.4 & 35.8 \\
Anime & JPEG\_q90 & 32.1 & 44.29 & 8.5 & 7.6 \\
Anime & PNG\_lossless & 10.1 & 99.00 & 38.1 & 43.5 \\
Photo & HKN\_q90\_444\_cfl1 & 1270.7 & 41.06 & 570.9 & 52.8 \\
Photo & JPEG\_q90 & 231.1 & 48.75 & 12.3 & 9.7 \\
Photo & PNG\_lossless & 1385.1 & 99.00 & 190.0 & 58.4 \\
UI & HKN\_q90\_444\_cfl1 & 480.1 & 45.01 & 494.9 & 39.3 \\
UI & JPEG\_q90 & 152.1 & 40.60 & 9.6 & 8.5 \\
UI & PNG\_lossless & 4.4 & 99.00 & 26.4 & 36.8 \\
\bottomrule
\end{tabular}

\end{table}

\section{考察}
現行実装では、lossless に関してカテゴリ依存性が大きい。
Photo では HKN が PNG より小さくなる一方、UI/Anime では PNG 優位が残る。
一方で lossy では CfL 修正後に視覚破綻が解消され、PSNR が安定した。

特に Photo ではデコード時間が課題である。
本稿の測定では、Photo 2枚の平均で
PNG decode が約 17.9 ms、HKN decode が約 45.4 ms となり、
速度面では PNG に劣後した。

今後の改善対象は次の 3 点を優先する。
\begin{itemize}
  \item Photo 向け CfL gate の厳格化(不要適用の削減)
  \item inverse DCT + dequant の SIMD 化(AVX2)
  \item token decode hot path の分岐削減とプロファイル駆動最適化
\end{itemize}

\section{再現手順}
以下をリポジトリルートで実行すると、表・図・PDFを再生成できる。
\begin{verbatim}
cd bench
../bench_png_compare | tee ../paper/results/lossless_png_compare_latest.txt
cd ..
python3 paper/repro/parse_lossless_png_compare.py
python3 paper/repro/run_lossy_jpeg_hkn_compare.py --runs 2
python3 paper/repro/generate_qualitative_figures.py
python3 paper/repro/generate_tex_tables.py
cd paper && latexmk -lualatex -interaction=nonstopmode paper_ja.tex
\end{verbatim}

\end{document}
